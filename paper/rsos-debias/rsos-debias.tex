%% Author_tex.tex
%% V1.0
%% 2012/13/12
%% developed by Techset
%%
%% This file describes the coding for rsproca.cls

\documentclass[]{rsos}%%%%where rsos is the template name


\usepackage[T1]{fontenc}
\usepackage[utf8]{inputenc}


% tightlist command for lists without linebreak
\providecommand{\tightlist}{%
  \setlength{\itemsep}{0pt}\setlength{\parskip}{0pt}}

% From pandoc table feature
\usepackage{longtable,booktabs,array}
\usepackage{calc} % for calculating minipage widths
% Correct order of tables after \paragraph or \subparagraph
\usepackage{etoolbox}
\makeatletter
\patchcmd\longtable{\par}{\if@noskipsec\mbox{}\fi\par}{}{}
\makeatother
% Allow footnotes in longtable head/foot
\IfFileExists{footnotehyper.sty}{\usepackage{footnotehyper}}{\usepackage{footnote}}
\makesavenoteenv{longtable}



%%%% *** Do not adjust lengths that control margins, column widths, etc. ***

%%%%%%%%%%% Defining Enunciations  %%%%%%%%%%%
\newtheorem{theorem}{\bf Theorem}[section]
\newtheorem{condition}{\bf Condition}[section]
\newtheorem{corollary}{\bf Corollary}[section]
%%%%%%%%%%%%%%%%%%%%%%%%%%%%%%%%%%%%%%%%%%%%%%%

\begin{document}


%%%% Article title to be placed here
\title{A systematic machine learning approach to quantifying the spatial extent of bias in human population data from mobile phones}

\author{
Carmen Cabrera$^{1}$,
Francisco Rowe$^{1}$}

\address{
  $^{1}$Geographic Data Science Lab, Department of Geography and Planning, University of Liverpool, Liverpool, United Kingdom.\\
  $^{}$}
%%%% Subject entries to be placed here %%%%
\subject{
Mobile phone data,
Human mobility,
Explainable AI,
Spatial analysis}

%%%% Keyword entries to be placed here %%%%
\keywords{
Mobile phone data,
Human mobility,
Explainable AI,
Spatial analysis}

%%%% Insert corresponding author and its email address}
\corres{
  Carmen Cabrera\\
  e-mail: \href{mailto:C.Cabrera@liverpool.ac.uk}{\nolinkurl{C.Cabrera@liverpool.ac.uk}}
}

%%%% Abstract text to be placed here %%%%%%%%%%%%
\begin{abstract}
The abstract text goes here. The abstract text goes here. The abstract text goes here. The abstract text goes here. The abstract text goes here. The abstract text goes here. The abstract text goes here. The abstract text goes here.
\end{abstract}
%%%%%%%%%%%%%%%%%%%%%%%%%%%

\providecommand{\EndFirstPage}{%
}

\maketitle

The main document should include:

Title (no more than 150 characters)

Author names and affiliations, and ORCID iDs where available

Abstract (no more than 200 words). (This will be used in reviewer
invitation emails so please think about how to describe your work to
make it easy for a potential reviewer to determine whether they would be
suitable.)

All main manuscript text. Ensure that all figures, tables and any
relevant supplementary materials are mentioned within the text
References

Acknowledgements and funding statement (ensure that you have included
grant numbers and the names of any funding providers)

Tables with captions Figure captions with credits where relevant

\newpage

\section{Introduction (FR)}\label{introduction-fr}

Traditional data streams, such as the census and surveys have been the
primary official source to provide a comprehensive representation of
national populations in countries worldwide. However, fast-paced
societal changes and emergency disasters, such as climate-induced
hazards and COVID-19 have tested and accentuated weaknesses in
traditional data systems \citep{green2021}. Traditional data systems often
provide data in infrequent and coarse temporal and geographical
resolutions \citep{rowe23-bigdata}. Generally they are expensive to maintain
and operate, and are slow taking months or years since they data are
collected to their release \citep{rowe23-bigdata}. Data collection from
climate- or conflict-impacted areas is generally unfeasible because of
restrictions due to high levels of insecurity and risk
\citep{iradukunda2025}. Yet, fast-paced societal changes require high
frequency, granular and up-to-date information to support real-time
planning, policy and decision making.

At the same time, we have seen the confluence of two diverging trends in
data availability. On the one hand, growing evidence of declining survey
response rates across many countries over the last 20 years is
accumulating \textbf{{[}REF{]}}. Dwindling numbers in surveys can represent
distorted picture of society \textbf{{[}REF{]}}. On the other hand, significant
advances in sensor technology, computational power, storage and digital
network platforms have unleashed a data revolution producing large
trails of digital trace data \textbf{{[}REF{]}}. These data are now routinely
collected and stored. They offer spatially granular, frequent and
instant information to capture and understand human activities at
unprecedentedly high resolution and scale, with the potential to produce
real-time actionable intelligence to support decision making \textbf{{[}REF{]}}.
Hence, national statistical offices are actively seeking to integrate
these data into their national data infrastructure \textbf{{[}REF{]}}.

Mobile phone data (MPD) collected via GPS- and IP-based technology have
become a prominent source of nontraditional data to monitor population
changes. Increasing usage of mobile services on smartphones and wearable
devices have resulted in the generation of large volumes of geospatial
data, offering novel opportunities to advance understanding of spatial
human behaviour, and thus revolutionise research, business and
government decision making and practices \citep{rowe23-bigdata}. MPD are now
a core component of the digital economy, creating new market
opportunities for data intelligence businesses, such as Cuebiq/Spectus,
Safegraph and Locomizer. They have been used to create critical evidence
to support policy making, prominently during the COVID-19 pandemic. In
research, MPD have been used to develop innovative approach to infer
mode of transport {[}REF{]}, monitor footfall changes {[}REF{]}, profile daily
mobility signatures {[}REF{]}, sense land use patterns {[}REF{]}, predict
socioeconomic levels {[}REF{]}, define urban extents {[}REF{]}, quantify tourism
activity {[}REF{]} and estimate migration and population displacement {[}REF{]}.

However, the use of MPD present major epistemological, methodological
and ethical challenges \citep{rowe23-bigdata}. A key unresolved challenge is
potential biases in MPAD compromising their statistical
representativeness and perpetuate social injustice {[}REF{]}. Biases reflect
societal digital and socioeconomic inequalities. Biases emerge from
differences in the access and use of the mobile phone applications used
to collect MPD \citep{wesolowski13-biases}. Only a fraction of the population
in a geographical area owns a smartphone, and even an smaller share
actively uses a specific mobile phone app. In the UK, for example, 98\%
of the adult population have a mobile phone and 92\% of this population
use a smartphone \citep{ofcom23}, but a smaller percentage actively use
Facebook (70\%) or Twitter (23\%) \citep{statista24}. Additionally, biases
emerge from differences in the access and use of digital technology
across population subgroups reflecting socioeconomic and demographic
disparities. For instance, wealthy, young and urban populations
generally have greater access and more intensively use of mobile phone
applications, and therefore tend to be over-represented in MPD {[}REF{]}.

The use of biased MPD can thus have major practical and societal
implications. If used uncorrected, MPD reproduce selective patterns of
smartphone ownership and application usage, rendering inaccurate or
distorted representations of human population activity. Such
representations disproportionately reflect behaviours of younger, urban
and higher-income users while underrepresenting marginalised or
less-connected groups. Distorted representations based on biased MPD can
thus misguide decision making, policy and planning interventions, and
thus amplify existing socio-economic disparities. In practice, existing
applications of MPD often use uncorrected population statistics derived
from MPD and have thus been constrained to offer a partial picture for a
limited segment of the overall population. Such data can only afford to
provide rough signals about the spatial distribution of (e.g.~spatial
concentration), trends (e.g.~increasing) and changes (e.g.~low to high)
in populations \citep{rowe22-sensing-ukraine}. They have cannot provide a
full representation of the overall population.

Efforts have been made to measure and assess biases in aggregate
population counts from digital data sources. Existing analyses typically
measure the extent of bias measuring the system-wide difference in the
representation of population counts from digital platforms and censuses.
To estimate the representation of digital data sources, the penetration
rate is computed as the active user base of a digital platform over the
census resident population. Existing analyses have thus been able to
established systematic gender, age and socio-economic biases in
population data obtained via API (or Application Programming Interface)
from social media platforms, such as Facebook and Twitter/X. However,
this approach requires information on the demographic and socio-economic
attributes of the collected sample and has focused on estimating biases
at the country level. Yet, these attributes are generally unavailable
for MPD, and biases may vary widely across subnational areas. What is
missing is an systematic approach to measure biases in population counts
from digital platforms, when population attributes are unknown, and
quantify the geographic variability in the extent of biases in these
data.

To address this gap, this paper aims to establish a standardised
approach to empirically measure the extent of biases in population data
derived from digital platforms, and identify their key underlying
contextual factors across subnational areas. We seek to address the
following research questions:

\begin{itemize}
\tightlist
\item
  What is the comparative extent of population coverage of digital
  sources relative to widely-used traditional surveys?
\item
  How systematic is the association between larger biases and the
  over-representation of rural, more deprived, child and elderly
  populations?
\item
  To what extent, are population data assembled from multiple
  applications versus single applications associated with lower bias?
\end{itemize}

Our approach proposes a statistical indicator of population coverage to
measure the extent of bias, and uses explainable machine learning to
identify key contextual factors contributing to spatial variations in
bias. Biases in digital trace data can emerge from multiple sources,
such as algorithmic changes, device duplication and geographic location
accuracy {[}REF{]}. We do not intend to identify these individual sources of
error. We focus on quantifying the extent of ``cumulative'\,' bias; that
is, the resulting bias from the accumulation of these error sources. We
use data collected from single and multiple mobile phone apps, and
compare their results. As outlined above, we test the extent to which
biases can be mitigated by leveraging information from multiple apps
encompassing a more diverse user population. Specifically, we use two
single-app (i.e.~Facebook and Twitter/X) and two multi-app providers
(i.e.~Locomizer and a European provider). We focus on the use of
aggregated population counts as this has become a common ethical and
privacy-preserving practice for companies to provide access to highly
sensitive data for social good.

Our study makes two key contributions. * Methodological contribution
i.e.~what we hope to achieve with our approach / quality assessment
framework ideas + start setting standards of good practice in the use of
MPD.\\
* Substantive contribution - systematic evidence identifying key
predictor of biases + do we find evidence of lower biases / greater
population coverage for multi-app better than single app?

\section{Data and methods}\label{data-and-methods}

{[}NOTE: I think that we need a paragraph describing and providing an
overview of the methodological strategy, including both data and
methods. Two points are particularly crucial to connect: (1) The use of
data from March 2021 for our assessment against census data; and (2) the
use of multiple data sources. We need to describe the idea of single-
and multiple-sourced app data. I wonder if we should include a table
listing their general attributes: advantages and limitations in terms of
their temporal and spatial coverage and resolution. This may not be the
place for the table but would be good to consider for the book if we
compared GPS data to other sources.{]}

\subsection{Data (CC)}\label{data-cc}

\subsubsection{Facebook}\label{facebook}

We use anonymised aggregate location data from Facebook app users who
have the location services setting turned on on their smartphone for the
UK, covering March 2021, the month when the 2021 UK Census was carried
out. We use the Facebook Population dataset created by Meta and accessed
through their Data for Good Initiative
(\url{https://dataforgood.facebook.com}). Prior to releasing the datasets,
Meta ensures privacy and anonymity by removing personal information and
applying privacy-preserving techniques \citep{maas2019} . Small-count
dropping is one of these techniques. A data entry is removed if the
population or movement count for an area is lower than 10. The removal
of small counts may mean that population counts in small sparsely
populated areas are not captured. A second technique consists in adding
a small undisclosed amount of random noise to ensure that it is not
possible to ascertain precise, true counts for sparsely populated
locations. Third, spatial smoothing using inverse distance-weighted
averaging is also applied is applied to produce a smooth population
count surface.

The Facebook Population dataset offers information on the number of
active Facebook users in a spatial unit at a given point in time. The
data is temporally aggregated into three daily 8-hour time windows (i.e.
00:00-08:00, 08:00-16:00 and 16:00- 00:00). In this work, we are
interested in capturing resident population, so we consider only data
corresponding to the time window corresponding to the night-time hours
(00:00-08:00).

Spatially, the Facebook Population dataset is aggregated according to
the Bing Maps Tile System developed by Microsoft (Microsoft). The Tile
System is a geospatial indexing system that partitions the world into
tile cells in a hierarchical way, comprising 23 different levels of
detail (Microsoft). At the lowest level of detail (Level 1), the world
is divided into four tiles with a coarse spatial resolution. At each
successive level, the resolution increases by a factor of two. The data
that we used are spatially aggregated into Bing tile levels 13. That is
about 4.9 x 4.9km at the Equator \citep{maas2019}.

We process Facebook Population data to enable comparison with UK census
data. Specifically, we take the average of daily Facebook Population
data over March 2021, the census month, and aggregate it into UK Local
Authority Districts (LADs) to align with the census data. This approach
is used to generate the figures and results in the body paper. In the
Supplementary Information, we test alternative approaches, including
averaging over a single week in March 2021, or performing the spatial
aggregation before temporal averaging. Our findings indicate that the
results remain robust regardless of the chosen approach

\subsubsection{Twitter}\label{twitter}

We use an anonymised, openly available, analysis-ready dataset of active
Twitter users in the UK. The data is derived from X (previously Twitter)
in the form of monthly active user counts residing across the UK
geography. The dataset is based on tweets from UK users \citep{wang2022}
collected via the Twitter Academic API. These tweets are either
geolocated at the time of posting or manually geocoded using a bounding
box provided by the Twitter Academic API, based on the IP address of the
posting device. The full dataset includes 161 million tweets from
February 2019 to December 2021; however, we focus on data from March
2021 to align with the 2021 UK Census. Users' Local Authority District
(LAD) of residence is identified using a frequency-based home-location
algorithm. Further details on the dataset's methodology can be found in
\citep{wang2022} .

While the X Academic API is no longer available to download data, but
existing and future projects offer an opportunity for research based on
X data. Global repositories of historical geolocated tweet data are
accessed through the Internet Archive (1996) and Harvard Geotweet
Archive (\url{https://gis.harvard.edu/data}). Despite these limitations, we
consider X data as it remains a key source of historical digital trace
data.

\subsubsection{Multi-app GPS data: source 1}\label{multi-app-gps-data-source-1}

We sourced data from a data analytics company that collects GPS location
data from around 26\% of smartphones in the UK. The raw data is collected
for individual anonymised devices, from numerous smartphone applications
where the users have explicitly granted location-sharing permissions.
The full dataset covers 7 days corresponding to the first week of April
for the UK, including X GPS records and X unique devices. While the
dates covered by dataset do not exactly coincide with the 2021 UK Census
dates, the alignment is close.

We process the data to estimate users' place of residence based on a
commonly used rule-based classification (e.g. \citep[@zhong24working]{wang2022}), which assumes that the residence of a device owner
corresponds to the location with the highest number of GPS records
during night hours (7 PM -- 7 AM). For a location to be classified as a
residence, it must account for more than 50\% of recorded nighttime
locations and be visited at least twice during the period of study. To
ensure consistency when comparing with other data sources, we aggregate
these residence locations at the Local Authority District (LAD) level.

\subsubsection{Multi-app GPS data: source 2}\label{multi-app-gps-data-source-2}

Our analysis includes a second source of analysis-ready GPS location
data, which is openly-available on GitHub
(\href{https://t.ly/dzlzB}{https://t.ly/dzlzB)}). This dataset has already
been processed to identify the home location of users accoring to the
methodology described in \citep{zhong24working}. The raw data is collected by
a UK-based data service company, which licenses mobile GPS data from 200
smartphone apps and applies pre-processing methods to ensure user
privacy and anonymity. The full dataset covers the UK in November 2021.
While this period does not exactly coincide with the 2021 UK Census, the
difference of less than a year is considered sufficiently close for our
analysis.

To ensure consistency across datasets, we further process the data by
aggregating it spatially from the Middle Layer Super Output Area (MSOA)
level to the Local Authority District Level (LAD).

\subsection{Methods}\label{methods}

Our proposed methodology consists of two stages aimed at quantifying two
types of biases: coverage biases and representational biases. Coverage
biases relate to the sample size of the dataset and refer to the
proportion of the population covered in the dataset. Representational
biases, arise from the demographic and socioeconomic characteristics of
the users who generate the digital trace data through specific
technologies.

The first stage of our methodology seeks to quantify coverage biases by
examining the variations in coverage across different spatial units. We
leverage the spatial granularity of digital trace data to analyse
coverage biases at more localised spatial scales. This allows us to
identify the extent to which different regions are represented within
the datasets, revealing any potential underrepresentation or
overrepresentation in specific locations.

The second stage seeks to quantify representational biases. To do this,
we leverage the spatial heterogeneity of coverage biases and model this
variation in terms of demographic and socioeconomic variables that
characterise local populations. This analysis allows us to identify
which specific demogrpahic and socioeconomic population attributes, such
as average income, education level or age composition, are more likely
to be associated with higher values of coverage bias, thus highlighting
which population groups tend to be underrepresented in different sources
of digital trace data.

\subsubsection{Measuring coverage bias (CCA)}\label{measuring-coverage-bias-cca}

We define a metric to quantify the magnitude of coverage bias in each
subnational area. This metric is based on the population coverage of the
dataset, which we compute as the ratio of the population captured
(sample size) by dataset \(D\), denoted as \(P_i^D\), to the total local
population of an area, \(P_i\). Formally, the coverage \(c_i\) is given by:
\begin{equation}
c_i = \dfrac{P_i^D}{P_i} \times 100,
\end{equation} where \(D\) identifies a given dataset, and \(i\) denotes
each subnational area. The resulting ratio \(c_i\) is assumed to take
values between \(0\) and \(100\), with 100 representing full population
coverage. If users have multiple accounts, the ratio can exceed \(100\),
since the total sample size could be greater than the local population
of area \(i\).

We then define the size of bias \(e_i\) as:

\begin{equation} \label{eq:size-bias}
e_i = 100 - c_i
\end{equation}

A value of \(e_i = 0\) indicates a lack of coverage bias, which
corresponds to full population coverage (\(c_i = 100\)). We use this bias
indicator to analysis the magnitude and spatial distribution of coverage
bias across multiple sources of digital trace data.

\subsubsection{Spatial patterns of bias (CC)}\label{spatial-patterns-of-bias-cc}

\subsubsection{Explainable machine learning (FR)}\label{explainable-machine-learning-fr}

We used explainable machine learning to identify the key predictors of
population bias and how these the importance of these predictors varies
across geographical areas. Existing evidence based on social media
suggests that population location data from digital platforms are biased
over-representing urban, wealthy and young-adult populations \textbf{{[}REF{]}}.
We therefore modelled our measure of population bias from
Equation\textasciitilde{}\ref{eq:size-bias} as a function of key area-level attributes
reflecting geographical differences in engagement and access to digital
technology across demographic, socioeconomic, household, housing and
location factors. \textbf{TABLE XX} reports the set of predictors included in
our analysis. We used data from the 2021 census for England and Wales to
measure these predictors.

We used an eXtreme Gradient Boosting (XGBoost) algorithm. XGBoost is an
ensemble that combines outputs from multiple models to produce a single
prediction and represents an efficient and scalable adaptation of the
gradient boosting machine algorithm proposed by \citep{friedman2001a}. It
utilises gradient descent to improve model performance, and decision
trees are built iteratively, with each tree built to minimise the error
residuals of a preceding iteration. XGBoost has been optimised for
scalability and computational efficiency, providing high predictive
accuracy with limited training time \citep{chen2016, nielsen2016tree}.
XGBoost has also become one of the most widely-used off-the-shelf
machine learning models in applied settings because of its built-in
regularization that mitigates overfitting, sparsity-aware tree
construction and parallelisation efficiency \citep{chen2016}. It can
accommodate nonlinearities and is robust to multicollinearity
\citep{chen2016}. We fitted the following XGBoost regression model.

\begin{equation} \label{eq:xgb-model}
\widehat{e}_i 
= \sum_{m=1}^M f_m\bigl(D_i, S_i, H_i, U_i, L_i\bigr),
\quad f_m \in \mathcal{F}
\end{equation}

\(e_i\) is our measure of population bias. \(f_m\) denotes an individual
regression tree from the boosted ensemble \(\mathcal{F}\) and \(M\) is the
total number of trees. The input variables \(D\), \(S\), \(H\), \(U\), \(L\)
represent key demographic, socioeconomic, housing, household, and
locational attributes of area \(i\), respectively. The model iteratively
learns the contribution of each feature to the prediction of the bias
indicator \(e_i\), allowing for complex, nonlinear interactions.

To implement Equation\textasciitilde{}\ref{eq:xgb-model}, we randomly split the data
into training (80\%) and testing (20\%) sets to ensure robust model
evaluation. We used 10-fold cross validation to train models and
performed grid search over learning rates, tree depths, subsample
ratios, and regularisation penalties to identify optimal
hyperparameters. We applied regularisation penalties including L1
(Lasso) and L2 (Ridge) terms to penalise overly complex trees, promote
feature sparsity, improve model generalisation and mitigate
multicollinearity among predictors. XGBoost's tree-based structure
additionally handles multicollinearity by hierarchically selecting the
most informative splits \citep{chen2016}. We then fitted a final model on the
full training set using these tuned settings of optimal parameters and
evaluated on the held-out test set. We evaluated models based on the
number of trees minimising the root mean squared error (RMSE), the
convergence of training and test error, and difference between predicted
and observed values.

\section{Results}\label{results}

\subsection{Varying extent of bias across data sources}\label{varying-extent-of-bias-across-data-sources}

As digital trace data becomes increasingly accessible, it opens up new
avenues for studying human behaviours with remarkable temporal and
spatial precision, extensive geographic coverage, and near real-time
access. However, the potential presence of biases can undermine the
validity of the data to deliver statistically representative evidence.

In this section, we focus on quantifying the biases in multiple sources
of digital trace data that arise due to the extent of population
coverage, i.e.~the proportion of the total population captured in the
dataset. In Figure , we contextualise these findings by comparing them
with various traditional datasets, particularly key UK surveys available
through the UK Data Service \citep{ukdataserviceSurveysData}. On the
\(x\)-axis, we represent two variables: at the top, the sample size of the
dataset, expressed as the number of respondents or subjects per 1,000
people, which reflects the population coverage of the dataset; and at
the bottom, a measure of bias in terms of this coverage, as defined in
equation \ref{eq:size-bias}. The figure highlights the remarkable
ability of digital trace data to capture a larger share of the total
population compared to traditional surveys, thanks to the automated,
passive nature of data collection on digital platforms. This contrasts
with the manual recruitment and data collection processes required for
surveys. As a result, the size of bias is generally lower for digital
trace data, highlighting its potential to inform comprehensive empirical
analyses.

\begin{figure}
\centering
\includegraphics{figures/compare-surveys-two-axis.png}
\caption{Size of bias and population coverage (per 1,000 population) by data
source.}\label{fig:survey}
\end{figure}

While the findings in Figure \ref{fig:survey} demonstrate the potential
of digital trace data compared to traditional data sources, high
population coverage alone does not ensure the data is representative of
different population groups.

In surveys, specific strategies are usually implemented during the data
generation process to improve the statistical representativeness of the
sample. For example, sampling techniques such as stratified sampling or
cluster sampling can be applied so that the sample reflects the broader
population of interest. After sampling, if certain groups remain
under-represented, responses can be adjusted using post-stratification
techniques. However, even when these strategies are applied, there is no
guarantee that the survey will be fully representative of the broader
population of interest \citep{cochran1977sampling}. This is because
representativeness can only be achieved with respect to a finite set of
attributes (e.g.~age, gender, income levels, location, etc.). Ensuring
perfect representativeness would only be possible either by surveying
the whole population.

With digital trace data, achieving statistical representativeness is
even more challenging. Unlike survey data, which is actively collected
using structured sampling methods, digital trace data is generated
passively as a byproduct of online interactions, transactions, or device
usage, without any control over who is included in the dataset.
Furthermore, by the time this data reaches researchers or analysts, it
is often anonymised, and does not contain demographic identifiers. As a
result, it is not possible to apply the standard post-stratification
weighting techniques that are typically used to adjust survey or census
data for improved representativeness.

We argue that, even though we do not always have specific demographic
information of the individuals captured through digital trace data, we
can infer some of these characteristics by leveraging the
spatio-temporal granularity of digital trace data. We argue that this is
a necessary first step to understand which population groups might be
under or over-represented in different sources of digital trace data.
This information is necessary to later adjust the data so that it is
more representative of the population of interest.

\subsection{The spatial distribution of biases (CC)}\label{the-spatial-distribution-of-biases-cc}

Next, we take advantage of the detailed geographic information in the
digital trace data to analyse bias at smaller, more localised spatial
levels. This helps us understand how well different geographic areas are
represented in the datasets. Since local populations vary in their
socioeconomic characteristics, we can use the degree of bias at these
smaller scales in the next step of our analysis, to determine which
population attributes are most associated with underrepresentation in
the data.

Figure \ref{fig:size-bias-spatial} shows the geographic variation of the
size of bias at the Local Authority District (LAD) level. Each row in
the figure corresponds to each of the digital trace datasets analysed
here. Within each row, we include: i) an hexagonal cartogram for the
size of bias in each LAD, representing the LADs as hexagons of equal
size to simplify the visualisation while maintaining relative positions;
with this cartogram, we report Moran's I as a measure of spatial
autocorrelation and its associated p-value, ii) a histogram of the size
of bias, showing the distribution of values across LADs, iii) a scatter
plot of the population covered by the digital trace data vs.~the actual
population of each LAD; with the scatter plot, we include the Pearson
correlation coefficient and its associated p-value.

\includegraphics[width=5.55208in,height=4.35417in]{figures/Fig-size-bias.png}

Examining the spatial variation in bias size, we observe distinct
patterns across the DT datasets considered. These varied spatial
patterns likely stem from differences in the demographic composition of
users for each technology. Factors such as age, socioeconomic status,
digital literacy, and regional preferences for certain platforms or
devices may contribute to these variations. Bias tends to display
stronger spatial patterns for Meta data and the second source of
multi-app GPS data, with lower bias in the North of England and Wales.
In contrast, Twitter/X data and the first souce of multi-app GPS data
follow more mixed patterns, as demonstrated by the values of Moran's I
closer to zero. Twitter/X data generally exhibits high bias, except in
London, the South East, and a few isolated areas. Similarly, bias in
first source of multi-app data tends to be lower in the South and South
East.

Turning to the histograms, we observe that bias size is highest for
Twitter/X data, with all values exceeding 99.5 except for a single
outlier, the City of London. This outlier likely arises due to the
unique demographic and occupational characteristics of the area. While
relatively few people reside in the City of London, it hosts a large
number of workers, including temporary professionals, who may be staying
in hotels. The home-detection algorithm in \citep{wang2022} used to generate
the Twitter/X data used here might classify the workplace or temporary
accommodations of City of London workers as their primary residences,
leading to an anomalously low bias measurement. Following Twitter/X, the
second source of multi-app GPS data exhibits the next highest bias
values. In contrast, the first source of multi-app data shows lower
bias, while Meta data has the lowest overall bias. Notably, Meta data
also displays the widest distribution of bias values, indicating greater
variability across different locations.

The scatter plots show a high linear correlation between the population
covered by the digital trace data and the actual population of each LAD,
as demonstrated by the Pearson coefficient, all above 0.8. This suggests
that, on average, the actual population in the LADs is not an indicator
of the size of bias in DT data, as the population coverage \(c_i\) remains
the consistent regardless of \(P_i\) . This could be a result of the fact
that the biases in the data are not driven by the number of people, but
rather by other their demographic characteristics such as age, income or
educational level. In the next section, we explore the variability of
demographic attributes of local populations as possible determinants of
the size of bias.

\subsection{Explaining biases (FR)}\label{explaining-biases-fr}

\section{Discussion (FR)}\label{discussion-fr}

\section{Conclusion (CC)}\label{conclusion-cc}

\ethics{Please provide details on the ethics.}

\dataccess{Please provide details on the data availability.}

\aucontribute{Please provide details of author contributions here.}

\competing{Please declare any conflict of interest here.}

\funding{Please provide details on funding}

\disclaimer{Please provide disclaimer text here.}

\ack{Please include your acknowledgments here, set in a single paragraph. Please do not include any acknowledgments in the Supporting Information, or anywhere else in the manuscript.}

\bibliographystyle{RS}
\bibliography{sample.bib}


\end{document}
